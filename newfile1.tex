%% LyX 2.3.7 created this file.  For more info, see http://www.lyx.org/.
%% Do not edit unless you really know what you are doing.
% \documentclass[english]{amsart}
\documentclass[12pt,reqno]{article}
% \usepackage[T1]{fontenc}
% \usepackage[latin9]{inputenc}
\usepackage{amstext}
\usepackage{amsthm}
\usepackage{amsmath}
\usepackage{xcolor}

\newtheorem{theorem}{Theorem}
\makeatletter
%%%%%%%%%%%%%%%%%%%%%%%%%%%%%% Textclass specific LaTeX commands.
% \numberwithin{equation}{section}
% \numberwithin{figure}{section}

\makeatother

% \usepackage{babel}
\begin{document}


\begin{abstract}
Abstract: find a function $x(t)$ that extremizes the functional $L$. Here we are going to outline the 
proof of the Euler Lagrange equation.
\end{abstract}


\section{Theorem Statement and Proof}\label{thm:EL}


\begin{theorem}
\textcolor{red}{Please state theorem precisely}
\end{theorem}

\begin{proof}


Let $x(t)$ be the function that extremizes the functional \textcolor{red}{What are the bounds for the integral?}$$L(x(t)) =
\int x(t)dt.$$ \textcolor{red}{Why does $x$ even exist.} Let $A(t)$ be the function that represents the divergence \textcolor{red}{what do you mean by divergence exactly?}
from the idealized path $x(t)$. Let $A(t_{1})=A(t_{2})=0$. All possible
paths can be represented by $x(t,a) = x(t) + aA(t)$ where a parameterizes
the divergence of the path. 

Let $$L(a) = \int_{t_{1}}^{t_{2}}f(x(t,a),\dot{x}(t,a),t)dt, $$ be functional
parameterized by $a$. 

Note when $a = 0$, $\frac{dL}{da}$=0. 

$\frac{dL}{da}=\frac{d}{da}\int_{t_{1}}^{t_{2}}f(x(t,a),\dot{x}(t,a),t)dt$

$0=\int_{t_{1}}^{t_{2}}\frac{\text{\ensuremath{\partial f}}}{\partial x}\frac{\partial x}{\partial a}+\frac{\text{\ensuremath{\partial f}}}{\partial\dot{x}}\frac{\partial\dot{x}}{\partial a}dt$

 
\begin{align*}
    \frac{\partial x}{\partial a} & =A(t)\\
    \frac{\partial\dot{x}}{\partial a} & =\frac{\partial x}{\partial a}\times\frac{d}{dt}=A'(t)
\end{align*}


\[
0=\int_{t_{1}}^{t_{2}}\frac{\text{\ensuremath{\partial f}}}{\partial x}\cdot A(t)+\frac{\text{\ensuremath{\partial f}}}{\partial\dot{x}}\cdot A'(t)
\]

Using Integration by parts: 
\begin{align*}
\int\frac{\text{\ensuremath{\partial f}}}{\partial\dot{x}}\cdot A'(t)dt & =\frac{\partial f}{\partial\dot{x}}A(t)\Vert_{t_{1}}^{t_{2}}-\int\frac{d}{dt}\cdot\frac{\text{\ensuremath{\partial f}}}{\partial\dot{x}}*A(t)dt\\
 & =-\int\frac{d}{dt}\cdot\frac{\text{\ensuremath{\partial f}}}{\partial\dot{x}}*A(t)dt
\end{align*}

Thus
\begin{align*}
0 & =\int_{t_{1}}^{t_{2}}\frac{\text{\ensuremath{\partial f}}}{\partial x}\cdot A(t)-\frac{d}{dt}\frac{\text{\ensuremath{\partial f}}}{\partial\dot{x}}\cdot A(t)dt\\
 & =\int A(t)\cdot(\frac{\partial f}{\partial x}-\frac{d}{dt}\cdot\frac{\partial f}{\partial\dot{x}})dt
\end{align*}

As $A(t)$ is arbitrary everywhere on the interval 
\begin{align*}
\frac{\partial f}{\partial x}-\frac{d}{dt}\cdot\frac{\partial f}{\partial\dot{x}} & =0\\
\frac{d}{dt}\cdot\frac{\partial f}{\partial\dot{x}} & =\frac{df}{dx}
\end{align*}

proving Euler's equation. This equation must be satisfied for a function
x(t) that extremizes the functional L. 

\section{Applications}
\subsection*{Arc-Length}

An immediate consequence of the Euler-Lagrange equation is the classical 
fact that the shortest distance between two points is a straight line. We will 
show this derivation now.
\begin{theorem}\label{thm:arc-length}
Let $a$ and $b$ be two real numbers with $a < b$. Let $c$
and $d$ be two arbitrary real numbers. Then the unique differentiable function $x(t)$ that extremizes the functional
\[
S(x)=\int_a^b \sqrt{1+(\frac{dx}{dt})^{2}}dt. 
\]
is the linear function $x(t) = mt + y_0$ for some real numbers $m$ and $y_0$.
\end{theorem}

Note that the integrand in Theorem~\ref{thm:arc-length} is precisely the arc-length formula 
from Calculus. Note we can solve for $m$ and $y_0$ \textcolor{red}{How?}.

Note that we call $S$ a functional since it takes in a function as the 
input and outputs a real number.
% This is the functional that calculates the arc length of a given path.

\begin{proof}

Let $x(t)$ be the function that extremizes $S(x)$. \textcolor{red}{Why does this exist?}

The integrand in $S$ is 
\[L(t,x,x') = \sqrt{1+(x')^{2}}.\]

We know from Theorem~\ref{thm:EL} (Euler Lagrange Equation) and the fact that $x$ minimizes $S$, that 
\[\frac{dL}{dx} = \frac{d}{dt}\frac{dL}{dx'}.\]
Since $x$ is not present on in $L(t,x,x')$, we have
\[\frac{dL}{dx} = 0.\]
So we have (\textcolor{red}{fix these equations})


\begin{align*}
    0 &= \frac{d}{dt}\frac{dL}{dx'} \\
    &=  \frac{d}{dt}\frac{-\dot{x}}{\sqrt{1-(\dot{x})^{2}}}\\
 & = \frac{-\ddot{x}\sqrt{1-(\dot{x})^{2}}+\dot{x}^{2}\ddot{x}(1-\dot{x})^{-\frac{1}{2}}}{1-(\dot{x})^{2}}\\
& = \frac{\ddot{x}}{(1-\ddot{x}^{2})^{\frac{3}{2}}}=0
\end{align*}
Thus 
$$\ddot{x}  =0 ,$$
on the interval $(a,b)$ \textcolor{red}{optional: why?}.
% \int\int\ddot{x}dt & =\int\int0\,dt\\
Therefore, from Calculus, we know there are 
constant $m$ and $y_{0}$ such that
$$
x  =mt+y_0
$$


\end{proof}

\subsection{Spring Pendulum}

Proving the shortest distance between two points is a straight line. 

Spring pendulum

\[
U=mg(l+x)cos(\theta)+\frac{1}{2}kx^{2}
\]

The Kinetic energy of the system is the linear kinetic energy of the
spring and the tangential Kinetic energy. 

\[
K=\frac{1}{2}m\dot{x}^{2}+\frac{1}{2}m(l+x)^{2}\dot{\theta}^{2}
\]

These energies are combined to make the Lagraganian
\[
L=K+U
\]

Let f(x) be the function that extremizes the action and thus is the
actual motion of the ball linearly. 

To find the function f(x) the following equation must be satisfied:
\begin{align*}
\frac{d}{dt}\frac{dL}{d\dot{x}} & =\frac{dL}{dx}\\
m\ddot{x} & =-mgsin(\theta)+kx+m(l+x)\dot{\theta}^{2}
\end{align*}

The RHS is the ma part of Newtons second law and the LHS is the gravitational
tangential force, the spring force, and the centrifugal force. 

Solving this, gives f(x). 

\[
\frac{d}{dt}\frac{dL}{d\dot{\theta}}=\frac{dL}{d\theta}
\]

\[
m(l+x)^{2}\ddot{\theta}=-mg(l+x)sin(\theta)
\]

This equation reveals that $\text{mass}$ $\times$tangential acceleration
= torque

\end{proof}




\end{document}
